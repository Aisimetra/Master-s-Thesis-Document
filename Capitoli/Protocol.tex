\newpage
\section*{Protocol}

\subsection*{First Part}
\begin{itemize}
\item 5 jumps
\item lie down on \textbf{prone} position
\item 5 jumps
\item lie down on \textbf{left side} position
\item 5 jumps
\item lie down on \textbf{supine} position
\item 5 jumps
\item lie down on \textbf{right side} position
\end{itemize}

\documentclass{article}

\usepackage[english]{babel}

\usepackage[letterpaper,top=2cm,bottom=2cm,left=3cm,right=3cm,marginparwidth=1.75cm]{geometry}

\usepackage{amsmath}
\usepackage{graphicx}
\usepackage[colorlinks=true, allcolors=blue]{hyperref}

\section{Data Collection Protocol}





The aim of this protocol is to collect data with the two mattresses (SensigTex under the Sensomative) of different heart rates and breath rates.
During this protocol will be asked to change positions to collect the data in all four positions: supine position, lateral right position, prone position, and lateral left position.
Before data collection will also record what the sensors measure in case there is not a participant on the mattress whether it is stationary or moving.




\section{Experimental Setting}
The experimental setting includes a bed above which two different sensors will be positioned: SensingTex and Sensomative. The Sensomative has to be placed under the  SensingTex Sensor and the sensor will be used simultaneously. 
\subsection{Mattress }
The Sensomative Sensor gives the best result for detecting breath rate and heart rate when placed under the chest.[\textbf{investigate possible distance from the start of the mattress}] For this reason, should be placed exactly []cm from the top part of the mattress.
Above this sensor, the SensingTex Sensor should be placed because applying even pressure will not affect the measurements.
\subsection{Pillow and bed sheets}

\subsection{Ground truth }
The ground truth value for this data collection will be collected via polysomnography and video cameras. Polysomnography allows to recording this following parameters:

\begin{itemize}
    \item Nasal flow and nasal pressure: that will be used as the ground truth for the breath rate.
    \item Chest and abdomen movement: that can be also used as ground truth for the breath rate, with different types of approaches. [\textbf{can also detect the heart rate o heart movement?}]
    \item SPO2 and Pulse with a fingertip: that will be the ground truth data for the heart rate
    \item \{ Raimon \} 
\end{itemize}

Along with Polysomnography will also be involved cameras to record the position of the participant for further analysis.

\section{Patient setting}
\subsection{what to wear}
The participant should wear comfortable clothes. T-shirts are recommended in order not to have a too high layer of fabric that could interfere with the detection of data, in particular with the heart bit detection.
\subsection{Lay down position}
There are four different positions that will be asked to\begin{itemize}
    \item Lateral left position
    \item Lateral right position
    \item Prone
    \item Supine
\end{itemize}
\subsection{Pattern Position}
One of the main points of this data collection is to find peaks in different positions, so each patient will be asked to lay down in a different position pattern.
\begin{itemize}
    \item supine position, lateral right position, prone position, and lateral left position
    \item lateral right position, prone position, lateral left position and supine position
    \item prone position, lateral left position, supine position and lateral right position
    \item lateral left position, supine position, lateral right position and prone position
\end{itemize}
Once a position pattern is given to a specific participant it has to be repeated for each step of the protocol.
\subsection{how long they have to lay down}





\section{Data Collection}
\subsection{Joint Data Collection}
\subsubsection{Breathing rate and Heart rate at rest in normal condition}
In this part of the protocol, the aim is to collect data on the breathing rate and heart rate of the participants at rest.
\subsubsection{Breathing rate and Heart rate at rest during mattress movement}
In this part of the protocol, the aim is to collect data on the participant's breathing rate and heart rate at rest while the mattress is moving.
\subsubsection{Breathing rate and Heart rate after an effort}
In this part of the protocol, the aim is to collect data on the participant's breathing rate and heart rate after an effort, to analyze data of an accelerated breath and heart bit and gradually decelerated.
The effort consisted of ten jumping jacks repeated before each position change.

%\section*{Breath Rate}



\subsection{Heart Rate Data Collection}
\subsubsection{Heart rate at rest with constrained breath rate}
In this part of the protocol, the aim is to provide a known rhythm to the breath rate according to an acoustic time. The sound will indicate when start to inhale and exhale to the participants.

\subsubsection{Heart rate at rest with constrained breath rate during mattress movement}
In this part of the protocol, the aim is to provide a known rhythm to the breath rate according to an acoustic time while the mattress is moving. The sound will indicate when start to inhale and exhale to the participants.

\section{Words}

