\chapter{Conclusion}
This thesis's projects have been focused to investigate the possibility of estimating a patient's respiratory rate using a sensor pressure mattress. 

In order to understand this possibility has been conducted a preliminary study of Sensomative mattress data, already available from previous data collection. This has highlighted the possibility, of using a multiresolution overlap discrete wavelet transform to denoise the data to recreate a clean wave to then apply a peak finder algorithm. Consequently having defined as breath the moment between inhaling and exhaling, has been possible to asset the respiration rate per minute as the number of peaks in a minute, retrieved from the signals of the mattress but also visible in the nasal pressure of the cardiopulmonary polysomnography.

The project concentrates on the SensingTex mattress whose data were not available and leading to the necessity to conduct a data collection. In the course of that, it has been recreated the condition to have variability in the data and also collected while the Somnomat Casa was moving. The second phase aims to understand if the movement of the rocking bed could influence the signals.

Seen the number of SensingTex sensors has been necessary to design a pipeline that discriminates the ones that represent a respiratory pattern from the others, through the definition of a metric that asset the percentage of confidence to have a respiratory pattern in that specific signal for that window. Within the pipeline is it possible to choose: the approach with whom use the criteria, binary or weighted, and the methods to denoise the raw data, such as multiresolution overlap discrete wavelet or Savitzky–Golay filter.
The best channels retrieved with the pipeline are analysed with a peak finder to asset the respiration rate per minute estimated from that sensor. The result is an average of the estimated respiratory rate per minute between the best sensors.

Looking at the mean average error of the result for each position, the Savitzky–Golay filter has a lower error in almost every position of 10\%. Although the error also seems to be determined by the position of the person on the mattress, in the prone position this may be due to different movements of the chest that is in direct contact with the bed. Even if with a not negligible error of average 2 breaths, the first aim of this thesis, i.e. to understand the feasibility to estimate respiratory rate from a sensor pressure mattress, has been addressed.

The data collected while the bed is moving are slightly worse in respect of the data collected on a normal bed. However, this increase is not as consistent as we would have expected. This knowledge leads to accomplishing the second goal of the data collection and of this project, i.e. to understand if the movement of the rocking bed affects the signal. There is a slight effect but not to be prevented from estimating the respiratory rate.
The objectives of the thesis are then achieved.

The limitation of this thesis is the small number of participants in the data collection, that are in the same age group even if well balanced between genders. Since different age groups have different respiratory rates at rest, exploring them could be significant to improve the pipeline. A further limitation is a mean average error that is too high to use this approach in the medical context, such as in the study of sleep stages which requires more accurate estimation. Although in domestic use, to satisfy a person's desire to track respiratory rhythm, this error is not a issue.