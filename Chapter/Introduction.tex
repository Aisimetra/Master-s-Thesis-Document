
\chapter{Introduction}

\begin{comment}
l'intro ha una sua struttura, in 3 parti (sigh ho già scritto questa 
storia decine di volte, ma mai in generale per darla poi come lettura 
a chi deve fare l'intro per la prima volta): da 0.5 a 1 pagina di super-short-intro, 
questa parte non ho citazioni biblio, inizia con la stringa "scopo di questo lavoro è" 
e dice solo cosa è stato fatto. 
la seconda parte è quella corposa (massomeno da 2 
a 5 pagine) e "gronda citrazioni" e descrive cosa è stato fatto dando anche il contesto
(che ovviamente non ci sta in 0.5 - 1 pagina). la terza parte è la solita cavolata 
dell'indice in forma discorsiva (nel capitolo x si descrive questo e quello..., e così via)

@online{sensomative,
howpublished = {\url{https://sensomative.com/}},
  note = "visited on dd.mm.yyyy"
  }
}

@online{sentex,
howpublished={\url{https://sensingtex.com}},
  note = "visited on dd.mm.yyyy"
  }
}
\end{comment}

%\subsection*{super-short-intro, inizia con lo scopo del lavoro è solo cosa è stato fatto}

This work aims to investigate the possibility of estimating a patient's respiratory rate using a sensor pressure mattress and how a rocking bed could influence its use. 
At first, it is focused on studying approaches to extract the breath and heart rate from pressure sensors using a dataset already available from previous studies. 
Then the work is focused on respiratory rate, and for this reason, it is conducted data collection using an innovative pressure textile-sensor mattress and a cardiorespiratory as ground truth: the primary objective is to collect data to understand the feasibility of extracting breath rate from the mat in case of stationary bed; the second goal is to understand if the movement of the rocking bed could influence the signal. 
In the second part, a pipeline to analyse the extracted data is created: from each mattress sensor, the signals are processed to exclude the ones without meaningful information and designed metrics that asset the confidence that from a sensor could be extracted a respiratory pattern. The remains signals are filtered to eliminate noise using multiresolution analysis of the maximal overlap discrete wavelet transform and Savitz-Golay filter to obtain a clean wave from which could be counted the number of breaths a person in a minute. As a result, the respiration rate per minute of the person is obtained and compared with the cardiopulmonary polysomnography to asset the error. The influence of the rocking bed on the mattress is obtained via the comparison of the mattress performance on the stationary bed. As a result of the pipeline is also available a heat-map to visualise where these best channels are positioned in respect of the body and so in the mattress.


\newpage


\subsubsection{Parte succosa}
Sleep is one of the most important physiological functions. Sleep quality can affect physical 
and mental wellness; for this reason, it is crucial to monitor vital signs and sleep stages without interfering with natural sleep. 
The state-of-the-art to monitor physiological data during sleep is polysomnography \cite{Penzel2016ModulationsPolysomnography}, which involves recording sleep stages, respiratory rate, heart and other parameters. However, this procedure is time-consuming, complicated, expensive,
 invasive for the patient and only sometimes available in hospitals. Even in its simplified version, cardiorespiratory polysomnography \cite{CallejaComparisonApnoea}, where are involved just nose cannulas, chest belts and electrodes for an electrocardiogram (ECG) and does not track neurophysiological variables, the patient is subjected to physical discomfort throughout the night.

Breathing monitoring is also crucial because inside the population it is present higher percentage of 
sleep-related breathing disorders that can be studied and monitored with this instrument, like sleep apnoea/hypopnoea syndrome (SAS), 
where the individuals experience a collapse 
of the airway in deeper sleep states. The ability to monitor it allows for a faster and closer intervention in severe cases. 

Also in the study of sleep stages, it is known that every phase and stage is characterised by different muscle tones, brain wave patterns, eye movements and heart and breathing rate alterations.
So if focused on one of the vital signs that characterise the different sleep stages like respiratory rate which in particular slowly becomes more stable in the Non-Rapid Eye Movement (NREM) phase and increases during the Rapid Eye Movement (REM) phase; this characterisation of the different stages gives the possibility to understand in which stage a person is based just on the respiratory signal.

x-----x


Nowadays, it is possible to achieve this goal using different unobtrusive methods, like pressure sensor mattresses.
They can be installed over the standard mattress and are now available as textile-sensor, which means that they can be as thin as possible and lead to less possible discomfort, but at the same time, can be used to track the respiratory rate and, depending on area density and sampling frequency, even hearth rate. For this reason, this thesis used two different kinds of pressure-sensor textiles mattresses:

The first, from \textit{Sensomative} \cite{sensomative}, with a higher sampling frequency and can cover a smaller area of a mattress, means that it needs to be positioned in a specific position and case the patient moves; it is possible not to have any more information. Previous studies have brought out the possibility of estimating breathing patterns and heart rates, so this thesis will explore this possibility.

The second, from \textit{SensingTex} \cite{sentex}, with a lower sampling frequency, is less expensive and with a total area that covers all mattresses; in addition, it is already installed in a hospital ward of the \textit{University of Bern} for the study research on movement disorders during sleep in patients with Parkinson’s disease. Therefore the ability to estimate breath and heart rate could be helpful in this study.

In the lab where this thesis is carried out is available a rocking bed (\textit{Somnomat}) aims to interact with the person and study how to improve sleep quality via vestibular stimulation. Also, in this case, the possibility of tracking vital signs could be significant, so the possibility of integrating the second mattress with the Somnomat is addressed.



\subsection*{citazioni, cosa è stato fatto con contesto e citazioni}


 % \cite{Bakker2021EstimatingSeverity}. 
%Respiration is also central in 
% one of the most common sleep disorders, sleep apnea, in this case, causes them to experience reduced time in stage N3 and REM sleep. si può every da 



\subsection*{indice in forma discorsiva \\ lo compilo come ultime cosa}

\subsection*{Acknowledgement}
The project is carried out in collaboration with \textit{Sensory-Motor System Lab} of Prof.~Robert Riener at \textit{Eidgenössische Technische Hochschule 
(ETH) Zürich} and supervised by Dr.~Alexander Breuss, Dr.~Oriella Gnarra and Dr.~Manuel Fujis.



\begin{comment}

\newpage
\begin{enumerate}
    \item  \textbf{Introduzione} \\  strutturato in 3 parti: 0.5 a 1 pagina di super-short-intro,descrive cosa è stato fatto dando anche il contesto (con citazioni) e riassunto di cosa ci sarà nei capitoli
    \item \textbf{Preliminaries} \\ contiene al suo interno tutta la parte di stato dell'arte dei seguenti argomenti e tutte le informaizoni necessarie per comprendere la tesi.
    \begin{enumerate}
        \item \textbf{Sleep Stages} \\ Il progetto nasce nel contesto di avere la necessita di poter monitorare la respirazione dei pazienti durante la notte, dalla letteratura si evince come la variazione nella respirazione possa essere usato per tracciarli.
        \item \textbf{Respiratory Rate} \\ Parlando di repirazione durante la tesi è necessario introdurre come funzioni la respirazione e come si definisca un respiro.
        \item \textbf{Cariorespiratory Polysomnography} \\ Come si effettuano ad oggi i montoraggi delle respirazione in ospedale, evidenzaindone i limiti.
        \item \textbf{Unobstrusive approaches} \\ citazione di altri sistemi usati ora, non ustrusivi e i motivi per cui non si vogliano usare nel nostro caso
        \item \textbf{Pressure Sensor Mattress} \\ stato ell'arte di cosa si possa fare con i materassi a pressione e similari
    \end{enumerate}
    \item \textbf{Methods} 
    \begin{enumerate}
        \item \textbf{Instrument} \\ strumenti coinvolti nella tesi
        \begin{enumerate}
            \item \textbf{SensingTex} \\ materasso grande brutto 10hz
            \item  \textbf{polisomnografia NOXA1} \\ utilizzata per tracciare la respirazione del soggeto
            \item \textbf{Somnomat} \\ letto che si muove, si vuole capire se sia utilizzabile insieme al materasso
        \end{enumerate}  
        \item  \textbf{Data Collection} \\ descrizione della necessità di avere dati per poter studiare la possibilitò di estrarre il ritmo respiratorio
        \begin{enumerate}
            \item \textbf{Normal Bed} \\ letto normale, le persone fanno 4 salti e poi si sdraiano in 4 posizioni diverse (totale 16 salti). serve per avere variabilità nei dati
            \item  \textbf{rocking bed, somnomat } \\ letto che si muove, persona sdraiata sopra che si gira in 4 posizioni, serve per vedere possibili alterazioni nei dati dovute dal movimento del letto
        \end{enumerate}  
    \end{enumerate}  
    \item  \textbf{Data Analysis} \\ Descrizione della pipeline, dai dati del materasso, preprocessamento, filtri vari, al numero di respiri al minuto per quel specifico minuto.
    \begin{enumerate}
        \item \textbf{Weighted and binary} \\ la pipeline viene creata sia dando un peso ai vari controlli effettuati sul segnale, sia rendendoli binari (o passa o non passa)
        \item  \textbf{Pipeline} \\ Descrizione effettiva della pipeline
        \begin{enumerate}
            \item \textbf{Excluding criteria} \\ criteri di esclusione dei canali, non si effettuano ulteriori analisi.
            \item  \textbf{SNR ratio} \\ deve rimanere in un intervallo, però avevo fatto casino inserendolo quindi non so se voglio citarlo
            \item  \textbf{Wavelet} 
            \begin{enumerate}
                \item teoria di come funziona
                \item applicazione nel progetto
            \end{enumerate} 
            \item \textbf{Savitz-Golay filter}
            \begin{enumerate}
                \item teoria di come funziona
                \item applicazione nel progetto
            \end{enumerate} 
            \item \textbf{Subsequent analyses of the filtered signa} \\ analisi sul segnale filtrato, controllo del numero di respiri, controllo della distanza tra picchi e valli che viene intesa sia come durata (distanza sull'asse del tempo in un intervallo +-20\%) oppure distanza euclidea picco valle (sempre +- 20\%).
        \end{enumerate} 
        \item \textbf{Result of the Pipeline (visual)} \\ risultati visuali ottenuti dalla pipeline. quidni la possibilità di visualizzare dove son i canali migliori e farci delle considerazioni.
    \end{enumerate} 
    \item \textbf{Result}
    \begin{enumerate}
        \item \textbf{Evaluation Metrics} \\spiegazione delle metriche utilizzate e la motivazione
        \begin{enumerate}
            \item Mean absolute error (MAE)
            \item Mean absolute percentage error (MAPE)
            \item Root Mean Square Error (RMSE)
        \end{enumerate} 
        \item \textbf{Result for Wavelet}
        \begin{enumerate}
            \item Bland-Altman plot 77durante la discussione su mi hanno detto che è meglio utilizzare questi per raprpesentare i risultati. ma avendo pochi dati non saprei
        \end{enumerate} 
        \item \textbf{Result for sg filter}
        \begin{enumerate}
            \item Bland-Altman plot
        \end{enumerate} 
        \item \textbf{Comparison between the two approaches (wavelet and SG filter)} \\ piccolo commento su una che va meglio dell'altro, sg filter >> wavelet
        \item  \textbf{Discussion performance on normal vs rocking bed)} \\ spiegre che funziona comnque a prescindere
    \end{enumerate} 
    \item \textbf{Conclusion and future discussin} \\ gli ampliamenti futuri ed un sunto sul fatto che si possa usare
\end{enumerate}
\end{comment}