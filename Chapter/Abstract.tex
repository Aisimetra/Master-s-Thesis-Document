\chapter*{Abstract}
Sleep is one of the most important physiological functions. Sleep quality can affect physical 
and mental wellness; for this reason,
 it is crucial to monitor vital signs without interfering with natural sleep. 
 The state-of-the-art to monitor physiological data during sleep is polysomnography \cite{Penzel2016ModulationsPolysomnography},that involves 

 which involves a cannula, chest belts and electrodes for ECG.

 During sleep, a person cycles between two phases Non-Rapid Eye Movement (NREM) and Rapid Eye Movement (REM);
  this second phase is further divided into three other stages (N1-N3). Different muscle tones, brain wave patterns, 
  eye movements, and heart and breathing rate alterations characterise every phase and stage.
\textbf{[explenation of sleep stages]}.
Focusing on one of the vital signs that characterise the different sleep stages is the respiratory rate 
which slowly becomes more stable going from the awake to the REM phase. Respiration is also central in 
one of the most common sleep disorders, sleep apnea, in this case, the individuals experience a collapse 
of the airway in deeper sleep states, causing them to experience reduced time in stage N3 and REM sleep. 
The ability to monitor it allows for a faster and closer intervention in severe cases. 

As said before, the state-of-art is a cumbersome device that 
requires cables attached to the users' bodies and often interferes with 
natural sleep. For this reason, we decided to use an unobtrusive sensor 
placed over the typical mattress to allow us to study the possibility of
 monitoring the patient's health. The sensor involved in this project is a 
 pressure-sensor textile from SensingTex [x], with a sensor area of 192 x 94 
 cm filled with 1056 sensors (sensor area density of 4 sensors for 10cm$^2$).

 The high number of sensors leads to the necessity of an algorithm to discriminate the ones from whom 
 it is possible to extract valuable information about the respiratory rate of the person that is on the mattress.
 Many of these channels are with null or stationary information; others present just interference from the 
 mattress itself. From just a few sensors, it is possible to retrieve a respiratory pattern and extract the 
 respiratory rate per minute (rpm). To obtain them from all the possibilities, we have defined our confidence 
 metric to assign each sensor a percentage of confidence that it will be a suitable sensor.
 \textbf{[parlare del somnomat]}.

 Since no data and polygraphs were available from the same person
 on the SensingTex was necessary to collect new data, this gives 
 us the possibility also to understand if this kind of Instrument
  could work with a lab project called Somnomat. A special 
  rocking bed \textbf{[parlare del somnomat e di come potrebbe essere integrato]}. The protocol involved 
  6 participants (three female /three male) between 20-30 years of age. It was divided into two phases: one 
  with the sensor mattress over a standard bed and the second on the rocking bed while this was moving. 
  Each participant had a fixed set of positions that had to be while lying on the mattress. 
  We also decide to insert variability into the data since, during sleep, respiratory rate increases in different stages. 
  In order to do that, we ask a participant to perform a set of five jumps before lying down in the first 
  phase of the protocol; for the second part, we ask them to turn around on the bed and stay on it 
  while it is moving without performing any activity.


  the data collection also involved Cardiorespiratory polysomnography. In our case, we use the nox a1 and noxturnal app. 
  \textbf{chennal selected and canulas}

  x---x

  the data coming from the data collection was cleaned in this methods, than was used this approach to smoot or filtering the singal.
  than was use this to asset the number of breath


