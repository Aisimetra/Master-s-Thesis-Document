\documentclass[a4paper,11pt, oneside,italian]{article}

\usepackage[utf8]{inputenc}
\usepackage{hyperref}
\usepackage{framed}
\usepackage{xcolor}
\usepackage{amsmath}
\usepackage{enumitem}
\usepackage{relsize}
\usepackage{microtype}
\usepackage{typearea}

\hypersetup{
  pdftitle = {Respiratory Rate Estimation using a Pressure Sensor Mattress}, 
  pdfauthor = {Artemisia Sarteschi},
  pdfsubject = {Riassunto della tesi},
  pdfpagemode = UseNone
}


\usepackage{fancyhdr}
% \pagestyle{fancy}
% \fancyhead[LE,RO]{\slshape \rightmark}
% \fancyhead[LO,RE]{\slshape \leftmark}
\fancyfoot[C]{\thepage}

\title{Respiratory Rate Estimation using a Pressure Sensor Mattress} 
\author{Artemisia Sarteschi\\\smaller matr.~829677}
\date{}
\makeatletter
\renewcommand{\paragraph}{%
  \@startsection{paragraph}{4}%
  {\z@}{0.75ex \@plus 1ex \@minus .2ex}{-1em}%
  {\normalfont\normalsize\bfseries}%
}
\makeatother

\def\SLP{\mbox{\rm {\sf SLP}}}
\def\rank{\mbox{\rm {\sf rank}}}
\def\lcs{\mbox{\rm {\sf lcs}}}
\def\lcp{\mbox{\rm {\sf lcp}}}
\def\lce{\mbox{\rm {\sf lce}}}
\def\LCE{\mbox{\rm {\sf LCE}}}
\def\ROT{\mbox{\rm {\sf ROT}}}
\def\select{\mbox{\rm {\sf select}}}
\def\col{\mbox{\rm {\sf col}}}
\def\NULL{\mbox{\rm {\sf null}}}
\def\len{\mbox{\rm {\sf len}}}
\def\pos{\mbox{\rm {\sf pos}}}
\def\row{\mbox{\rm {\sf row}}}
\def\LF{\mbox{\rm {\sf LF}}}
\def\FL{\mbox{\rm {\sf FL}}}
\def\W{\mbox{\rm {\sf w}}}
\def\A{\mbox{\rm {\sf A}}}
\def\A{\mbox{\rm {\sf A}}}
\def\SA{\mbox{\rm {\sf SA}}}
\def\LCP{\mbox{\rm {\sf LCP}}}
\def\ISA{\mbox{\rm {\sf ISA}}}
\def\PLCP{\mbox{\rm {\sf PLCP}}}
\def\RLCP{\mbox{\rm {\sf RLCP}}}
\def\RLBWT{\mbox{\rm {\sf RLBWT}}}
\def\MEM{\mbox{\rm {\sf MEM}}}
\def\KMEM{\mbox{\rm {\sf K-MEM}}}
\def\KSMEM{\mbox{\rm {\sf K-SMEM}}}
\def\MS{\mbox{\rm {\sf MS}}}
\def\LCP{\mbox{\rm {\sf LCP}}}
\def\NSV{\mbox{\rm {\sf NSV}}}
\def\PSV{\mbox{\rm {\sf PSV}}}
\def\RMQ{\mbox{\rm {\sf RMQ}}}
\def\BWT{\mbox{\rm {\sf BWT}}}
\def\BWM{\mbox{\rm {\sf BWM}}}
\def\ITR{\mbox{\rm {\sf index\_to\_run}}}
\def\GS{\mbox{\rm {\sf get\_symbol}}}
\def\PBWT{\mbox{\rm {\sf PBWT}}}
\def\MPBWT{\mbox{\rm {\sf mPBWT}}}
\def\MRLPBWT{\mbox{\rm {\sf mRLPBWT}}}
\def\DPBWT{\mbox{\rm {\sf dPBWT}}}
\def\RLPBWT{\mbox{\rm {\sf RLPBWT}}}
\def\SMEM{\mbox{\rm {\sf SMEM}}}
\def\M{\mbox{\rm {\sf M}}}
\def\C{\mbox{\rm {\sf C}}}
\def\Occ{\mbox{\rm {\sf Occ}}}
\def\L{\mbox{\rm {\sf L}}}
\def\F{\mbox{\rm {\sf F}}}
\def\DA{\mbox{\rm {\sf DA}}}
\def\DM{\mbox{\rm {\sf DM}}}
\def\PA{\mbox{\rm {\sf PA}}}
\def\LCE{\mbox{\rm {\sf LCE}}}
\def\UP{\mbox{\rm {\sf update}}}
\def\lceb{\mbox{\rm {\sf lce\_bounded}}}
\def\RM{\mbox{\rm {\sf reverse\_map}}}


\begin{document}
\maketitle
\setlist{leftmargin = 2cm}
\noindent


Sleep is one of the most important physiological functions. Sleep quality can affect physical 
and mental wellness; for this reason,
it is crucial to monitor vital signs without interfering with natural sleep. 
The state-of-the-art to monitor physiological data during sleep is polysomnography%\cite{Penzel2016ModulationsPolysomnography}
, which involves recording sleep stages, respiratory rate, heart and other parameters. However, this procedure is time-consuming, 
complicated, expensive, invasive for the patient and only sometimes available in hospitals. For this reason, it is nowadays 
acceptable to use cardiorespiratory polysomnography that does not track neurophysiological variables. This type of polysomnography
involves a cannula, chest belts and electrodes for an electrocardiogram (ECG) but does not involve an electroencephalogram (EEG).
Another reason why this kind of instrument is widely used is that inside the population, we have a higher percentage of 
sleep-related breathing disorders that can be studied and monitored with this instrument, like sleep apnoea/hypopnoea syndrome (SAS), where the individuals experience a collapse 
of the airway in deeper sleep states. The ability to monitor it allows for a faster and closer intervention in severe cases. 
The sleep cycle of a person is divided into two phases Non-Rapid Eye Movement (NREM) and Rapid Eye Movement (REM);
this second phase is further divided into three other stages (N1-N3). Different muscle tones, brain wave patterns, 
eye movements, and heart and breathing rate alterations characterise every phase and stage.
%\textbf{[explenation of sleep stages]}.
Focusing on one of the vital signs that characterise the different sleep stages is the respiratory rate 
which slowly becomes more stable going from the awake to the REM phase; this characterisation of the different stages gives the possibility to 
understand in which stage a person is based just on the respiratory signal.
 % \cite{Bakker2021EstimatingSeverity}. 
%Respiration is also central in 
% one of the most common sleep disorders, sleep apnea, in this case, causes them to experience reduced time in stage N3 and REM sleep. si può every da 
As said before, the state-of-art is a cumbersome device that 
requires cables attached to the users' bodies and often interferes with 
natural sleep. 

[parte sulle altre cose e le cose di privacy]


\subsection*{Instruments}
For this reason, this thesis aims to study the possibility to use an unobtrusive sensor placed over the usual mattress to retrieve 
respiratory rate without discomfort for the person lying down on it. 


The sensors in this project appear like a thin mattress similar in size to a common one that can be easily installed with adjustable straps.
In particular, the sensors are pressure-sensor textiles from \textit{SensingTex®}; in our case, was used the Pressure Mat Dev Kit,
 that has a sensor area of 192 x 94 cm filled with 1056 sensors (hereafter also referred to as "Channels") sampled at 250hz
 with a total sensor area density of 4 sensors for 10cm$^2$.
 The raw data extracted from the mattress can be viewed together to visually see the position of the person since the sensors are pressure sensors
the different pressures exerted by the presence/absence of a body on it or by its parts are given as a number inside an interval. 
So it is possible to create a heat map (or heatmap) to show the variation in colour of the intensity of the pressure, which can create the shape of
a person on the mattress.

Looking closer into signals of singles channels is possible to see a pattern that resembles a breathing rhythm,  similar to the data that can
 be retrieved from the nasal pressure exerted on the cannula of cardiorespiratory polysomnography.
This pattern was the key factor in deciding to use this sensor mattress (hereafter also referred to as "Sensor Mat" or "Mat"). 
In the laboratory where this project was carried on, was available a rocking bed (Somnomat) involved in a study of an intervention for 
sleep apnea, it was decided to address another question or if it is possible to retrieve the respiratory rate while the rocking bed is moving.
The possibility of integrating SensingTex® with Somnomat could be significant to have a closer and faster intervention on sleep apnea.

\subsection*{Data Analysis Pipeline}

The total number of sensors is 1056 and consequently the same number of signals from the mattress.
this leads to the necessity of an algorithm to discriminate the ones from whom 
it is possible to extract valuable information about the respiratory rate of the person on the mattress.
Many of these channels are with null or stationary information; others present just interference from the 
mattress itself. From just a few sensors, it is possible to retrieve a respiratory pattern and extract the 
respiratory rate per minute (rpm). Therefore becomes necessary to design a metric that underlines these channels.

x-----x \\

x----X \\


The meaning of this metric must be interpreted as confidence expressed as the goodness of the signal in percentual.




\textbf{[parlare del somnomat]}.

\subsection*{Data Collection}
Since no data and polygraphs were available from the same person
on the SensingTex was necessary to collect new data, this gives 
us the possibility also to understand if this kind of Instrument
could work with a lab project called Somnomat. A special 
rocking bed \textbf{[parlare del somnomat e di come potrebbe essere integrato]}. The protocol involved 
6 participants (three female /three male) between 20-30 years of age. It was divided into two phases: one 
with the sensor mattress over a standard bed and the second on the rocking bed while this was moving. 
Each participant had a fixed set of positions that had to be while lying on the mattress. 
We also decide to insert variability into the data since, during sleep, respiratory rate increases in different stages. 
In order to do that, we ask a participant to perform a set of five jumps before lying down in the first 
phase of the protocol; for the second part, we ask them to turn around on the bed and stay on it 
while it is moving without performing any activity.


the data collection also involved Cardiorespiratory polysomnography. In our case, we use the nox a1 and noxturnal app. 
\textbf{chennal selected and canulas}

x---x

the data coming from the data collection was cleaned in this methods, than was used this approach to smoot or filtering the singal.
than was use this to asset the number of breath

\subsection*{Result}

\subsection*{Acknowledgement}
The project is carried out in collaboration with \textit{Sensory-Motor System Lab} of Prof.~Robert Riener at \textit{Eidgenössische Technische Hochschule 
(ETH) Zürich} and supervised by Dr.~Alexander Breuss, Dr.~Oriella Gnarra and Dr.~Manuel Fujis.


\end{document}


% LocalWords:  pangenoma naive sottostringa BWT sottostringhe Durbin prefix MEM
% LocalWords:  array matching threshold divergence PBWT SMEM query statistics
% LocalWords:  RLBWT aplotipi aplotipo Burrows Wheeler RLPBWT maximal exact LCP
% LocalWords:  LCE