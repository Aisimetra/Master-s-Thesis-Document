\documentclass[a4paper,11pt, oneside,italian]{article}

\usepackage[utf8]{inputenc}
\usepackage{hyperref}
\usepackage{framed}
\usepackage{xcolor}
\usepackage{amsmath}
\usepackage{enumitem}
\usepackage{relsize}
\usepackage{microtype}
\usepackage{typearea}
\usepackage{hyperref}

\hypersetup{
  pdftitle = {Respiratory Rate Estimation using a Pressure Sensor Mattress}, 
  pdfauthor = {Artemisia Sarteschi},
  pdfsubject = {Riassunto della tesi},
  pdfpagemode = UseNone
}

\usepackage{fancyhdr}
% \pagestyle{fancy}
% \fancyhead[LE,RO]{\slshape \rightmark}
% \fancyhead[LO,RE]{\slshape \leftmark}
\fancyfoot[C]{\thepage}

\title{Respiratory Rate Estimation using a Pressure Sensor Mattress} 
\author{Artemisia Sarteschi\\\smaller matr.~829677}
\date{}
\makeatletter
\renewcommand{\paragraph}{%
  \@startsection{paragraph}{4}%
  {\z@}{0.75ex \@plus 1ex \@minus .2ex}{-1em}%
  {\normalfont\normalsize\bfseries}%
}
\makeatother

\def\SLP{\mbox{\rm {\sf SLP}}}
\def\rank{\mbox{\rm {\sf rank}}}
\def\lcs{\mbox{\rm {\sf lcs}}}
\def\lcp{\mbox{\rm {\sf lcp}}}
\def\lce{\mbox{\rm {\sf lce}}}
\def\LCE{\mbox{\rm {\sf LCE}}}
\def\ROT{\mbox{\rm {\sf ROT}}}
\def\select{\mbox{\rm {\sf select}}}
\def\col{\mbox{\rm {\sf col}}}
\def\NULL{\mbox{\rm {\sf null}}}
\def\len{\mbox{\rm {\sf len}}}
\def\pos{\mbox{\rm {\sf pos}}}
\def\row{\mbox{\rm {\sf row}}}
\def\LF{\mbox{\rm {\sf LF}}}
\def\FL{\mbox{\rm {\sf FL}}}
\def\W{\mbox{\rm {\sf w}}}
\def\A{\mbox{\rm {\sf A}}}
\def\A{\mbox{\rm {\sf A}}}
\def\SA{\mbox{\rm {\sf SA}}}
\def\LCP{\mbox{\rm {\sf LCP}}}
\def\ISA{\mbox{\rm {\sf ISA}}}
\def\PLCP{\mbox{\rm {\sf PLCP}}}
\def\RLCP{\mbox{\rm {\sf RLCP}}}
\def\RLBWT{\mbox{\rm {\sf RLBWT}}}
\def\MEM{\mbox{\rm {\sf MEM}}}
\def\KMEM{\mbox{\rm {\sf K-MEM}}}
\def\KSMEM{\mbox{\rm {\sf K-SMEM}}}
\def\MS{\mbox{\rm {\sf MS}}}
\def\LCP{\mbox{\rm {\sf LCP}}}
\def\NSV{\mbox{\rm {\sf NSV}}}
\def\PSV{\mbox{\rm {\sf PSV}}}
\def\RMQ{\mbox{\rm {\sf RMQ}}}
\def\BWT{\mbox{\rm {\sf BWT}}}
\def\BWM{\mbox{\rm {\sf BWM}}}
\def\ITR{\mbox{\rm {\sf index\_to\_run}}}
\def\GS{\mbox{\rm {\sf get\_symbol}}}
\def\PBWT{\mbox{\rm {\sf PBWT}}}
\def\MPBWT{\mbox{\rm {\sf mPBWT}}}
\def\MRLPBWT{\mbox{\rm {\sf mRLPBWT}}}
\def\DPBWT{\mbox{\rm {\sf dPBWT}}}
\def\RLPBWT{\mbox{\rm {\sf RLPBWT}}}
\def\SMEM{\mbox{\rm {\sf SMEM}}}
\def\M{\mbox{\rm {\sf M}}}
\def\C{\mbox{\rm {\sf C}}}
\def\Occ{\mbox{\rm {\sf Occ}}}
\def\L{\mbox{\rm {\sf L}}}
\def\F{\mbox{\rm {\sf F}}}
\def\DA{\mbox{\rm {\sf DA}}}
\def\DM{\mbox{\rm {\sf DM}}}
\def\PA{\mbox{\rm {\sf PA}}}
\def\LCE{\mbox{\rm {\sf LCE}}}
\def\UP{\mbox{\rm {\sf update}}}
\def\lceb{\mbox{\rm {\sf lce\_bounded}}}
\def\RM{\mbox{\rm {\sf reverse\_map}}}


\begin{document}
\maketitle
\setlist{leftmargin = 2cm}
\noindent


This work aims to investigate the possibility of estimating a patient's respiratory rate using a sensor pressure mattress and how a rocking bed could influence its use. 
At first, it is focused on studying approaches to extract the breath and heart rate from pressure sensors using a dataset already available from previous studies. Then the work is focused on respiratory rate, and for this reason, it is conducted data collection using an innovative pressure textile-sensor mattress: the primary objective is to collect data to understand the feasibility of extracting breath rate from the mat; the second goal is to understand if the movement of the rocking bed could influence the signal. 
In the second part, a pipeline to analyse the extracted data is created: from each mattress sensor, the signals are processed to exclude the ones without meaningful information and designed metrics that asset the confidence that from a sensor could be extracted a respiratory pattern. The remains signals are filtered to eliminate noise using multiresolution analysis of the maximal overlap discrete wavelet transform and Savitz-Golay filter to obtain a clean wave from which could be counted the number of breaths a person in a minute. As a result, the respiration rate per minute of the person is obtained, and a heatmap is used to visualise where these channels are positioned in respect of the body and so in the mattress.




\subsubsection*{Reason for the thesis (?) \\ penso ci voglia un minimo di divisorio per inserire la parte di cosa si sia fatto}
Sleep is one of the most important physiological functions. Sleep quality can affect physical 
and mental wellness; for this reason, it is crucial to monitor vital signs and sleep stages without interfering with natural sleep. 
The state-of-the-art to monitor physiological data during sleep is polysomnography \cite{Penzel2016ModulationsPolysomnography}, which involves recording sleep stages, respiratory rate, heart and other parameters. However, this procedure is time-consuming, complicated, expensive, invasive for the patient and only sometimes available in hospitals. Even in its simplified version, cardiorespiratory polysomnography, where are involved just nose cannulas, chest belts and electrodes for an electrocardiogram (ECG), the patient is subjected to physical discomfort throughout the night.
Nowadays, it is possible to achieve this goal using different unobtrusive methods, like pressure sensor mattresses.
They can be installed over the standard mattress and are now available as textile-sensor, which means that they can be as thin as possible and lead to less possible discomfort, but at the same time, can be used to track the respiratory rate and, depending on area density and sampling frequency, even hearth rate. For this reason, this thesis used two different kinds of pressure-sensor textiles mattresses:

The first, from \textit{Sensomative} \cite{sensomative}, with a higher sampling frequency and can cover a smaller area of a mattress, means that it needs to be positioned in a specific position and case the patient moves; it is possible not to have any more information. Previous studies have brought out the possibility of estimating breathing patterns and heart rates, so this thesis will explore this possibility.

The second, from \textit{SensingTex}\cite{sentex}, with a lower sampling frequency, is less expensive and with a total area that covers all mattresses; in addition, it is already installed in a hospital ward of the \textit{University of Bern} for the study research on movement disorders during sleep in patients with Parkinson’s disease. Therefore the ability to estimate breath and heart rate could be helpful in this study.

In the lab where this thesis is carried out is available a rocking bed (\textit{Somnomat}) aims to interact with the person and study how to improve sleep quality via vestibular stimulation. Also, in this case, the possibility of tracking vital signs could be significant, so the possibility of integrating the second mattress with the Somnomat is addressed.

\subsection*{citazioni, cosa è stato fatto con contesto e citazioni}


 % \cite{Bakker2021EstimatingSeverity}. 
%Respiration is also central in 
% one of the most common sleep disorders, sleep apnea, in this case, causes them to experience reduced time in stage N3 and REM sleep. si può every da 



\subsection*{indice in forma discorsiva}

\subsection*{Acknowledgement}
The project is carried out in collaboration with \textit{Sensory-Motor System Lab} of Prof.~Robert Riener at \textit{Eidgenössische Technische Hochschule 
(ETH) Zürich} and supervised by Dr.~Alexander Breuss, Dr.~Oriella Gnarra and Dr.~Manuel Fujis.

\bibliographystyle{ieeetr}

\bibliography{/Users/cucciolo/Desktop/Master-s-Thesis-Document/references.bib}

\end{document}

